\documentclass{article} % For LaTeX2e
\usepackage{nips, times}
\usepackage{hyperref}
\usepackage{url}
\usepackage{graphicx}
\usepackage{amsmath}
\usepackage{amssymb}
\usepackage{booktabs}
\usepackage{tabularx}
\usepackage{caption}
\usepackage{subcaption}
\usepackage{color}
\usepackage{relsize}
\usepackage{placeins}
\usepackage{natbib}
\usepackage{paralist}
\usepackage[toc,page]{appendix}

\renewcommand{\null}{\operatorname{null}}
\newcommand{\given}{\,|\,}

\input math.tex

\nipsfinalcopy

\begin{document}
\title{Image Segmentation for ISH embryos}

\author{
Fanny Yang\\
University of California\\
Berkeley, CA 94720, USA\\
\texttt{fanny-yang@eecs.berkeley.edu} \\
\And
Siqi Wu\\
University of California\\
Berkeley, CA 94720, USA\\
\texttt{siqi@stat.berkeley.edu}\\
}

\maketitle

\section{Previous works}

\paragraph{Deformable models}
Active contours:
\cite{Cootes92_ActiveShape,Cootes01_ActiveApp, Cootes92_TrainingShape, Kass88_Snakes}
\cite{Sclaroff01_RegionGroup, ElBaz09_ShapeApp, Leventon00_ShapeGeodesic}
Review papers: \cite{McInerney96_MIReview, Baswaraj12_ACReview}

Level based:
\cite{Tsai03_MILevelSets, ChanVese01}
Review paper \cite{Cremers07_Levelreview}

Level based with shape, intensity prior:
Cremers \cite{Cremers06_KernelDensity, Chan05_LevelShape, Chen09_LevelShapeIntensity}

\paragraph{Graphical models}
MRF: CRF: \cite{Lafferty01_CRFSeq, He04_MultiScale}
GraphCut: \cite{Lempitsky_BranchMin}
Sth completely different: \cite{Andres11_Probabilistic}

\paragraph{Hybrid models}
\cite{Huang04_MRFDM, Chen12_MIGraphCut, Uzunbas13_MultiOrgan, Schlesinger13}


\section{Model}
We are using a MRF closely linked to Huang. As a difference however we do not use contours $c$ which carry the burden of parameterization. But we use the continuous counterpart: level set functions, or more specifically, signed distance functions $\phi(i)$.

Let's denote $y$ as our hidden labels, $x$ as our observed image. We index pixels by $i$ and our signed distance function is $\phi$.
Our graphical model looks like in Huang, and thus our joint distribution with potentials $\psi$ reads
\begin{align}
p(x,y,\phi) &=  \prod_i \psi(x_i,y_i)  \prod_{i,j} \psi(y_i,y_j)\prod_i p(y_i|\phi)p(\phi) \label{total_joint}\\
&=: \prod_i p(x_i|y_i)p(y_i,y_j)p(y_i|\phi)p(\phi)\nonumber
\end{align}

We model $p(x_i|y_i)$ as Gaussians with parameters $\mu_{y_i}$ and $\sigma_{y_i}$, i.e. 
\beqs
p(x_i|y_i) = \frac{1}{\sqrt{2\pi}\sigma_{y_i}}\exp(-\frac{(x_i-\mu_{y_i})^2}{\sigma_{y_i}^2})
\eeqs

Our pairwise potentials 
\beqs
p(y_i,y_j) \propto \exp(-c|y_i-y_j|)/\theta_1 \Indi_{y_i\neq y_j} + \theta_2 \Indi_{y_i=y_j}
\eeqs

For the consistency measure of the labels $y$ on $\phi$ we use the logistic function
\beqs
p(y_i = 1|\phi) = \frac{1}{1 + \E^{-\lambda \phi(i)}}
\eeqs
where $\phi(i)$ is basically a signed distance function (possibly other distances) and $\lambda$ controls the sharpness of the slope. Note: Schlesinger uses $\psi(y|\phi) = \E^{\sum_i y_i \phi(i)}$.


\subsection{Level Set segmentation}
The term $p(\phi)$ enables us to incorporate priors and regularizations. In our model we use length and area as in \cite{Cremers06_KernelDensity, ChanVese01, MumfordShah89}. Note that instead of $p(x|\phi)$ in level set formulations we now have $p(x|y)$ and $p(y|\phi)$.
\begin{align*}
-\log p(\phi|\tilde{\phi}) &= E_{int}(\phi) + E_{prior}(\phi)\\
&= \lambda_1 \int_{\Omega}\delta_\epsilon(\phi(x,y))\|\nabla \phi(x,y)\|\d x \d y + \lambda_2 \int_{\Omega}H_\epsilon(\phi(x,y))\d x \d y\\
&- \log \sum_{i=1}^N \exp\left(-\frac{1}{2\sigma^2}d^2(H(\phi),H(\phi_i))\right)
\end{align*}
where $H$ is the Heaviside function and $\sigma^2$ is the width of the Gaussian kernel. $E_{int}$ denotes the regularization on length and area of the region (boundary), $E_{prior}$ is the energy given by the shape prior which is measured as a distance $d^2(H(\phi),H(\phi_i)) = \int_{\Omega}(H(\phi)-H(\phi_i))^2\d x$ is basically doing a nonparametric kernel density estimation for the shape distribution using given samples $\phi_i$. We choose for the kernel width $\sigma^2=\frac{1}{N}\sum_{i=1}^N \min_{j\neq i}d^2(H(\phi_i),H(\phi_j))$.


The task is now to maximize $\log Q_D(\phi|b) \propto \log p(b_i|\phi)p(\phi)$ which is equivalent to minimizing the energies $E_{int}, E_{prior}$. By the multivariate Euler Lagrange equations we know that $F(x) = \int \L(x,\phi(x),\nabla \phi(x))\d x$ is minimized if 
% for which Euler-Lagrange equation for multivariates $x$ yields the following fixed point equation for the optimal $\phi^*$:
\beqs
\frac{\partial}{\partial \phi} \L(x,\phi(x),\nabla \phi(x))  + \divop \frac{\partial}{\partial \nabla \phi} \L(x,\phi(x),\nabla \phi(x))=0
\eeqs 

We now add time for the evolution of the function and minimize the energy with respect to $\phi$ using the update equation
\begin{align}
\frac{\partial \phi}{\partial t} &= -\frac{\partial E_{int}(\phi)}{\partial \phi} - \frac{\partial E_{prior}(\phi)}{\partial \phi}  \label{labelset_update}\\
&=  \delta_{\epsilon}(\phi) (\lambda_1 \left[\divop(\frac{\nabla \phi}{\|\nabla \phi\|})\right] +\lambda_2) - \frac{\sum_i \alpha_i \frac{\partial}{\partial \phi} d^2(H(\phi),H(\phi_i))}{2\sigma^2 \sum_i \alpha_i}\nonumber
\end{align}
with $\alpha_i = \exp(-\frac{d^2(H\phi,H\phi_i)}{2\sigma^2})$.
This is basically doing a gradient descent. For the second term since $E_{prior}$ is a direct function of $\phi$, we can readily take the partial derivative with respect to $\phi$. $\frac{\partial}{\partial \phi} d^2(H(\phi),H(\phi_i))}
$ is a Gateaux derivative and yields
\beqs
\frac{\partial}{\partial \phi} d^2(H(\phi),H(\phi_i)) = \delta_{\epsilon}(\phi)(H(\phi)-H(\phi_i))
\eeqs

The question is how we can discretize the continuous first term in \eqref{labelset_update}. If we expand it we obtain
\beqs
\divop \frac{\nabla \phi}{\|\nabla \phi\|} = \frac{ \frac{\partial^2 \phi}{\partial x^2} (\frac{\partial \phi}{\partial y})^2 - 2 \frac{\partial^2 \phi}{\partial x \partial y} \frac{\partial \phi}{\partial x}\frac{\partial \phi}{\partial y} +  \frac{\partial^2 \phi}{\partial y^2} (\frac{\partial \phi}{\partial x})^2}{(\frac{\partial \phi}{\partial x}^2  + \frac{\partial \phi}{\partial y}^2)^{\frac{3}{2}}}
\eeqs
In our implementation we simply approximate the continuous derivatives by finite differences (this is exactly the formula they use in the Chan Vese Matlab package).

The overall formula for the update thus reads
\begin{align*}
\frac{\partial \phi}{\partial t} &= \delta_{\epsilon}(\phi)(\lambda_1  \frac{ \frac{\partial^2 \phi}{\partial x^2} (\frac{\partial \phi}{\partial y})^2 - 2 \frac{\partial^2 \phi}{\partial x \partial y} \frac{\partial \phi}{\partial x}\frac{\partial \phi}{\partial y} +  \frac{\partial^2 \phi}{\partial y^2} (\frac{\partial \phi}{\partial x})^2}{(\frac{\partial \phi}{\partial x}^2  + \frac{\partial \phi}{\partial y}^2)^{\frac{3}{2}}} + \lambda_2) \\
&- \frac{\sum_i \alpha_i  \delta_{\epsilon}(\phi)(H(\phi)-H(\phi_i))}{2\sigma^2 \sum_i \alpha_i}\nonumber
\end{align*}


\paragraph{Notes}
The hard things are the length and also the derivative of the Heaviside function, which is a $\delta$ distribution. Instead of getting a very sharp boundary we introduce a mollified continuous Heaviside $H_{\epsilon} = $.

\subsection{Variational inference}
The inference task is to maximize $p(y,\phi|x) \propto p(x,y,\phi)$. However finding the maximum of \eqref{total_joint} exactly is computationally intractable. We use structural variational inference \cite{Jordan99_Variational, Wainwright_Variational} to approximate the problem, in which we look for a $Q(y,\phi|x,a,b)$ with a certain structure and parameters $a,b$ and minimize the Kullback Leibler divergence $KL(Q||P) = \sum_\phi \sum_y Q(y,\phi|x,a,b) \log \frac{Q(y,\phi|x,a,b)}{p(y,\phi|x)}$
Our ansatz for the structure of $Q$ is the following
\beqs
Q(y,\phi|x,a,b) = Q_M(y|x,a)Q_D(\phi|b) 
\eeqs
with $Q_M(y|x,a) = \prod_{i,j} \psi(y_i,y_j) \prod_i  \xi(x_i,y_i) p(y_i|a_i)$ and $Q_D(\phi|b) \propto \prod_i p(b_i|\phi) p(\phi)$ so that we have factorized the distribution with respect to the two unknown vectors $y, \phi$. We are hence assuming independence of the two, given $a,b$ respectively.

We now need to minimize the KL divergence for which we follow the easy derivation in Blei:
Using the independence we can write out the KL 
\begin{align*}
&\EE \log Q_M(y|x,a) + \EE \log Q_D(\phi|b) - \EE \log p(y,\phi|x)\\
 = &\EE \left[\log Q_M(y|x,a)|\phi \right] + \EE \left[ \log Q_D(\phi|b) |y\right] - \EE \left[\EE \log p(y|\phi) |\phi \right]- \EE \log p(\phi)\\
= &\EE_{Q_M} (\log Q_M(y|x,a) - \EE_{Q_D} \log p(y|\phi)) + \EE_{Q_D} (\log Q_D(\phi|b) - \log p(\phi))\\
= &\EE_{Q_M} (\log Q_M(y|x,a) - \EE_{Q_D} \log p(y|\phi)) + \EE_{Q_D} (\log p(b|\phi) + c)
\end{align*}
where the first equality follows from the conditional independence of $y,\phi$ given $a,b$ and the second uses $Q_D(\phi) = p(\phi)$.

By Lagrange equations of first kind to incorporate the constraint of $\EE Q_D(\phi|b) |b = 1$ and $\EE Q_M(y|x,a) |x,a = 1$  we obtain that the Lagrange multiplier $\lambda = 1$ and the following must hold for optimal $a,b$:
\begin{align*}
& -\EE_{Q_D} \log p(y|\phi) + \log Q_M(y|x,a) = 0\\
\implies  &\sum_i (\EE_{Q_D}\log p(y_i|\phi) - \log p(y_i|a_i)) = \sum_{j\in N(i)} \log p(y_i,y_j) - \EE_{Q_D} \log(p(y_i,y_j)) \\
\implies &\log p(y_i|a_i) = \EE_{Q_D} \log p(y_i|\phi) 
\end{align*}
and similarly
\begin{align*}
&- \EE_{Q_M} \log p(y|\phi) + \log p(b|\phi) = 0\\
\implies &\log p(b_i|\phi) = \EE_{Q^i_M} \log p(y_i|\phi)  
\end{align*}
where $Q_M^i = Q_M(y_i|x,a)$.

These are the fixed point equations and we will aim at solving them iteratively using alternating minimization. 

\subsection{Algorithm}

We now turn to the algorithm pipeline
. 
\paragraph{Algorithm}
Initialize level set function $\phi$.
Step $k$.
\begin{enumerate}
\item HOW DO WE do this first step - we only cover a region of $\phi$ for the steps?
\item Calculate $\phi^* = \argmax_{\phi} Q_D(\phi|b)$ using the current $p(b|\phi)$ and approximate by $Q'_D(c|b) = \delta(\phi^*)$
\item Calculate $p(y_i|a_i) = \exp(\EE_{Q'_D} \log (y_i|a_i))= p(x_i|\phi^*)$.
\item In order to calculate $Q_M(y_i|x_i,a)$ estimate the parameters of $p(x_i|\theta_i)$ which are $\mu_1,\mu_0,\sigma_1,\sigma_0$ using the latent variable $y_i$ and the EM algorithm
\item Calculate $p(b_i|\phi) = \EE_{Q_M^i} \log p(y_i|\phi)$.
\end{enumerate} 

\subsection{EM algorithm}
We now turn to the step of estimating the parameters of the Gaussians $p(x_i|y_i)$ for $Q_M(y|x,a)$ for which we will use an EM algorithm. The aim is to calculate 
\beqs
\max_{\theta} p(x|\theta,a) = \sum_y p(x,y|\theta) = \sum_y p(x|y,\theta)p(y|a)
\eeqs
For the EM algorithm we have the E Step
\begin{align*}
\L(q^{(t+1)},\theta) &= \sum_y p(y|x,\theta^{(t)}) (\log p(x|y,\theta) + \log p(y|a)) \\
&= \sum_i \sum_{y_i} \log p(x_i|y_i,\theta) \sum_{y_j:j\neq i} p(y|x,\theta^{(t)},a) + c(\theta)\\
%&= \sum_i \sum_{y_i} p(y_i|x, \theta^{(t)},a) \log p(x_i|y_i,\theta) + c(\theta)\\
&= \sum_i \sum_{y_i} p(y_i|x, \theta^{(t)},a) \Indi_{y_i=l} (-\log \sigma_l - \frac{(x_i -\mu_l)^2}{2\sigma_l^2}) + c(\theta)\\
&= \sum_i \sum_l p(y_i = l|x,\theta^{(t)},a) (-\log \sigma_l -  \frac{(x_i -\mu_l)^2}{2\sigma_l^2}) + c(\theta)
\end{align*}

For the maximization step we maximize $\L(q^{(t+1)},\theta)$ over $\theta$ so that we have
\begin{align*}
\mu_l^{(t+1)} &= \frac{\sum_i p(y_i=l|x,\theta^{(t)},a) x_i}{\sum_i p(y_i=l|x,\theta^{(t)})}\\
\sigma_l^{2(t+1)} &= \frac{\sum_i p(y_i =l | x,\theta^{(t)},a)(x_i-\mu_l)^2}{\sum_i p(y_i = l|x,\theta^{(t)},a)}
\end{align*}
which intuitively just computes the average intensity/variance within one label.

\subsection{Belief Propagation}
As we see we need to compute the marginal likelihood 
\beqs
p(y_i|x,\theta^{(t)},a) \propto \sum_{y_j:j\neq i} \prod_i p(x_i|y_i) \prod_{i,j} p(y_i,y_j) \prod_i p(y_i|a_i)
\eeqs
This can be readily computed using a loopy belief propagation algorithm/junction tree algorithm (slow?).
%% We write down the complete likelihood
%% \beqs
%% p(x_i|\theta_i) = p(y_i=1|a_i)p(x_i|y_i=1,\theta_1) + p(y_i=0|a_i)p(x_i|y_i=0,\theta_0)
%% \eeqs
%% We need to maximize this likelihood. 


\subsection{}

\FloatBarrier
\vskip 0.2in
\nocite{*}
\bibliographystyle{plain}
\bibliography{lit}

\begin{appendices}

\end{appendices}
\end{document}
